\newpage
\section{Simulation Analysis}
\label{sec:simulation}

\subsection{Operating point analysis}
This section covers the circuit simulation in ngspice. However, for this simulation, 
a new node 8 and an auxiliary independent voltage source - $V_8$ - with a voltage of 0V had to be created. 
Node 8 is located between the resistance $R_6$ and node 3, while the terminals of $V_8$ are connected to nodes 3 and 8. 
This happens because, to create a current dependent voltage source, ngspice needs the current value of a voltage source. 
As the current $I_c$ doesn't flow through any voltage source, one had to be created. \\
\noindent
The values obtained, by ngspice, for currents flowing in each resistance (Ampers) and nodes voltages (Volts) are showed in the following table:
\begin{table}[h!]
  \centering
  \begin{tabular}{|c|c|}
    \hline    
    {\bf Name} & {\bf Value [A or V]} \\ \hline
    @cb[i] & 0.000000e+00\\ \hline
@ce[i] & 0.000000e+00\\ \hline
@q1[ib] & 7.022567e-05\\ \hline
@q1[ic] & 1.404513e-02\\ \hline
@q1[ie] & -1.41154e-02\\ \hline
@q1[is] & 5.765392e-12\\ \hline
@rc[i] & 1.411536e-02\\ \hline
@re[i] & 1.411536e-02\\ \hline
@rf[i] & 7.022567e-05\\ \hline
@rs[i] & 0.000000e+00\\ \hline
v(1) & 0.000000e+00\\ \hline
v(2) & 0.000000e+00\\ \hline
base & 2.254108e+00\\ \hline
coll & 5.765392e+00\\ \hline
emit & 1.411536e+00\\ \hline
vcc & 1.000000e+01\\ \hline

  \end{tabular}
  \caption{Operating point. A variable preceded by @ is of type {\em current}
    and expressed in Ampere; other variables are of type {\it voltage} and expressed in
    Volt. (g in "gib" refers to the Ngspice notation of a current source controlled by a voltage).}
  \label{tab:op}
\end{table}

\begin{table}[h!]
  \centering
  \begin{tabular}{|c|c|}
    \hline    
    {\bf Name} & {\bf Value [A or V]} \\ \hline
    zout & 7.866317e+00\\ \hline

  \end{tabular}
  \caption{Operating point. A variable preceded by @ is of type {\em current}
    and expressed in Ampere; other variables are of type {\it voltage} and expressed in
    Volt. (g in "gib" refers to the Ngspice notation of a current source controlled by a voltage).}
  \label{tab:op2}
\end{table}





