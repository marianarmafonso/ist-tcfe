\newpage
\section{Conclusion}
\label{sec:conclusion}
In order to analysis the exactness or discrepancies in the theoretical analysis and the simulation, the following tables made with ngspice and octave were presented.
\begin{table}[h!]
  \centering
  \begin{tabular}{|c|c|}
    \hline    
    {\bf Name} & {\bf Value [A or V]} \\ \hline
    @cb[i] & 0.000000e+00\\ \hline
@ce[i] & 0.000000e+00\\ \hline
@q1[ib] & 7.022567e-05\\ \hline
@q1[ic] & 1.404513e-02\\ \hline
@q1[ie] & -1.41154e-02\\ \hline
@q1[is] & 5.765392e-12\\ \hline
@rc[i] & 1.411536e-02\\ \hline
@re[i] & 1.411536e-02\\ \hline
@rf[i] & 7.022567e-05\\ \hline
@rs[i] & 0.000000e+00\\ \hline
v(1) & 0.000000e+00\\ \hline
v(2) & 0.000000e+00\\ \hline
base & 2.254108e+00\\ \hline
coll & 5.765392e+00\\ \hline
emit & 1.411536e+00\\ \hline
vcc & 1.000000e+01\\ \hline

  \end{tabular}
 \begin{tabular}{|c|c|}
 \hline
 \centering
    {\bf Name} & {\bf Value [A or V]} \\ 
    \hline
$I_c$ & 0 \\
$I_b$ & \partialinput{8}{8}{tabela1.tex} \\
$R_1[i]$  & \partialinput{9}{9}{tabela1.tex}\\
$R_2[i]$   & \partialinput{10}{10}{tabela1.tex} \\
$R_3[i]$ & \partialinput{11}{11}{tabela1.tex} \\
$R_4[i]$  & \partialinput{12}{12}{tabela1.tex} \\
$R_5[i]$ & \partialinput{13}{13}{tabela1.tex}\\
$R_6[i]$   & \partialinput{14}{14}{tabela1.tex} \\
$R_7[i]$ & \partialinput{15}{15}{tabela1.tex} \\
$V_1$           & \partialinput{1}{1}{tabela1.tex} \\
$V_2$  & \partialinput{2}{2}{tabela1.tex}\\
$V_3$   & \partialinput{3}{3}{tabela1.tex} \\
$V_5$  & \partialinput{4}{4}{tabela1.tex} \\
$V_6$   & \partialinput{5}{5}{tabela1.tex} \\
$V_7$    & \partialinput{6}{6}{tabela1.tex} \\
$V_8$     &  \partialinput{7}{7}{tabela1.tex}\\
\hline
 \end{tabular}
 \caption{Simulation (right) and Theoretical (left) Node Voltages and Branch currents t < 0.}
  \label{tab:conc1}
\end{table}
\newpage
\begin{table}[h!]
  \centering
  \begin{tabular}{|c|c|}
    \hline    
    {\bf Name} & {\bf Value [A or V]} \\ \hline
    zout & 7.866317e+00\\ \hline

  \end{tabular}
 \begin{tabular}{|c|c|}
 \hline
 \centering
    {\bf Name} & {\bf Value [A or V]} \\ 
    \hline
$I_b$ & \partialinput{11}{11}{tabela2.tex} \\
$R_1[i]$  & \partialinput{12}{12}{tabela2.tex}\\
$R_2[i]$   & \partialinput{13}{13}{tabela2.tex} \\
$R_3[i]$ & \partialinput{14}{14}{tabela2.tex} \\
$R_4[i]$  & \partialinput{15}{15}{tabela2.tex} \\
$R_5[i]$ & \partialinput{16}{16}{tabela2.tex}\\
$R_6[i]$   & \partialinput{17}{17}{tabela2.tex} \\
$R_7[i]$ & \partialinput{18}{18}{tabela2.tex} \\
$V_1$(t=0)  & 0\\
$V_2$(t=0)  & \partialinput{1}{1}{tabela2.tex}\\
$V_3$(t=0)   & \partialinput{2}{2}{tabela2.tex} \\
$V_5$(t=0)   & \partialinput{3}{3}{tabela2.tex} \\
$V_6$(t=0)    & \partialinput{4}{4}{tabela2.tex} \\
$V_7$(t=0)     & \partialinput{5}{5}{tabela2.tex} \\
$V_8$(t=0)     &  \partialinput{6}{6}{tabela2.tex}\\
\hline
 \end{tabular}
 \caption{Simulation (right) and Theoretical (left) Node Voltages and Branch currents t = 0.}
  \label{tab:conc2}
\end{table}

\noindent Comparing the theoretical and the simulation values, they are very almost the same only differing in the number of decimal places. Nonetheless, these approximation differences are negligible, regarding the low complexity of the circuit in analysis. Furthermore, the results obtained in the graphs obtained through ngspice and octave are also very similar.
\noindent Taking all this into account, the theorical analysis showed coherent values relative to the simulation, proving that the theoretical analysis of this circuit is accurate.


